\documentclass[a4paper,12pt]{scrreprt}
\usepackage[T1]{fontenc}
\usepackage[utf8]{inputenc}
\usepackage[ngerman]{babel}
\usepackage[table]{xcolor}% http://ctan.org/pkg/xcolor
\usepackage{tabu}
\usepackage{graphicx}
\usepackage{lmodern}

\begin{document}


%\titlehead{Kopf} %Optionale Kopfzeile
\author{Alexander Rieppel} %Zwei Autoren
\title{Transportlogistik} %Titel/Thema
\subject{Betriebs- und Informationsmanagement} %Fach
\subtitle{Ausarbeitung} %Genaueres Thema, Optional
\date{\today} %Datum
\publishers{5AHITT} %Klasse

\maketitle
\tableofcontents


\chapter{Einleitung}
\section{Allgemein}
	Der Begriff Transportlogistik beschreibt die Lieferung und Beförderung von Gütern, zwischen verschiedenen Orten innerhalb von Logistiknetzwerken (Transportnetzstruktur). Dieser Teilbereich der Logistik ist durch die starke Differenzierung der Transportdienstleister, durch umweltpolitische Bestrebungen sowie vereinzelt durch Projekte der Stadt- oder Citylogistik entstanden.\\\\
	Der Transport wird dabei in der Regel durch einen Spediteur durchgeführt. Die Transportorganisation übernimmt der Spediteur oder ein Logistikdienstleister, der meist zusätzliche Leistungen rund um die Transportlogistik anbietet.\\\\
	
	\chapter{Güterverteilzentrum}
	\section{Allgemein}
	\section{Standort}
	Standortplanung von Verteilzentren
	\chapter{Transportnetzwerke}
	\section{Allgemein}
	ERP4.pdf - Grundstruktur von Transportnetzwerken (Anfang)
	ERP5.pdf - Materialwirtschaft Logistik
	Manhattan Metrik
	Französische Eisenbahnmetrik
	Supply-Chain Management
	
\chapter{Quellen}

\end{document}