\documentclass[a4paper,12pt]{scrreprt}
\usepackage[T1]{fontenc}
\usepackage[utf8]{inputenc}
\usepackage[ngerman]{babel}
\usepackage[table]{xcolor}
% http://ctan.org/pkg/xcolor
\usepackage{tabu}
\usepackage{graphicx}
\usepackage{lmodern}
\usepackage{hyperref}
\usepackage{geometry}
\geometry{verbose,a4paper,tmargin=20mm,bmargin=20mm,lmargin=30mm,rmargin=30mm}



\begin{document}


%\titlehead{Kopf} %Optionale Kopfzeile
\author{Alexander Rieppel} %Zwei Autoren
\title{Transportlogistik} %Titel/Thema
\subject{Betriebs- und Informationsmanagement} %Fach
\subtitle{Ausarbeitung} %Genaueres Thema, Optional
\date{\today} %Datum
\publishers{5AHITT} %Klasse
{\Huge }

\maketitle
\tableofcontents


\chapter{Einführung}
	Der Begriff Transportlogistik beschreibt die Lieferung und Beförderung von Gütern, zwischen verschiedenen Orten innerhalb von Transportnetzwerken. Dieser Teilbereich der Logistik ist durch den heutigen Wohlstand entstanden, in dem sich Unternehmen auf ihre Kernkompetenzen konzentrieren und darüber hinausgehende Dienstleistungen und Produkte europaweit und international einkaufen. Die Planung und Optimierung, Ausführung, Überwachung und Steuerung der damit verbundenen Güterströme ist Aufgabe der Transportlogistik. 
	Neben dem Transport von Waren als solches wird auch die Be- und Entladung zur Transportlogistik gezählt. \\\\ 
	Das Problem Güter von A nach B zu bringen ist für jeden Unternehmer eine Herausforderung. Prinzipiell ist es dabei egal, ob die Ware per LKW, Zug, Schiff, Flugzeug, Pipeline, etc. transportiert wird. Es geht rein darum, dass die Ware ankommt, sie möglichst unbeschadet ankommt und vor allem schnell am Ziel ist. Auch der Kostenfaktor spielt dabei eine große Rolle und auch ökologische Nachhaltigkeit, gewinnt zunehmend an Bedeutung.\\\\
	Obwohl es im Prinzip egal ist womit transportiert wird, ist der am meisten genutzte Verkehrsträger zur Güterbeförderung, speziell hierzulande, die Straße. Zu den entscheidenden Gründen zählt die Netzbildungsfähigkeit des LKW, der jede Quelle und Senke flexibel erreichen kann. Die Stärke von Schiene und Binnenwasser-straße liegen in dem effizienten Transport von Massengütern über längere Strecken. Für die besonders langen Distanzen im internationalen Gütertransport werden das Seeschiff für große Volumina und das Flugzeug eher für besonders eilige oder wertvolle Fracht genutzt. Neben der reinen Beförderung als offensichtlichste transportlogistische Funktion erfordern die Ver- und Entsorgung von Industrie- und Handelsunternehmen weitere
	Leistungen, wie die Lagerhaltung für die zeitliche Überbrückung zwischen Fertigung und
	Absatz, den Umschlag im Rahmen des Verkehrsmittelwechsels und die Kommissionierung
	zur Vereinzelung nachgefragter Mengen. Hinzu kommen verstärkt auch Tätigkeiten, die
	einen zusätzlichen Mehrwert am Gut schaffen, wie die Montage von Teilen zu Modulen,
	die Aufarbeitung von Produkten, ihre Etikettierung oder Sequenzierung.\\\\
	Die vorhergehenden Ausführungen zeigen, dass die Gütertransportlogistik steigenden
	Anforderungen unterliegt, durch einen stetigen Anstieg gekennzeichnet ist, aber auch eine
	Fülle von Gestaltungsoptionen bietet. Dies belegt die Bedeutung der Transportlogistik mit der Folge, dass sich Gesellschaft, Wirtschaft und Wissenschaft intensiv mit Fragestellungen auf diesem Gebiet auseinander setzen müssen.
	
	\chapter{Güterverkehr}
	
	
	\chapter{Transportnetzwerke}
	\section{Einführung}
	ERP4.pdf - Grundstruktur von Transportnetzwerken (Anfang)
	
	\chapter{Güterverteilzentrum}
	\section{Allgemein}
	
	\section{Standortplanung}
	Standortplanung von Verteilzentren
	
	\chapter{Wegplanung}
	\section{Manhattan-Metrik}
	Manhattan Metrik\\
	ERP5.pdf - Materialwirtschaft Logistik
	\section{Französische Eisenbahnmetrik}
		Französische Eisenbahnmetrik

	
\chapter{Quellen} 
\nolinkurl{http://de.wikipedia.org/wiki/Transportlogistik}

\end{document}