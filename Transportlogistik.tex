\documentclass[a4paper,12pt]{scrreprt}
\usepackage[T1]{fontenc}
\usepackage[utf8]{inputenc}
\usepackage[ngerman]{babel}
\usepackage[table]{xcolor}% http://ctan.org/pkg/xcolor
\usepackage{tabu}
\usepackage{graphicx}
\usepackage{lmodern}
\usepackage{hyperref}

\begin{document}


%\titlehead{Kopf} %Optionale Kopfzeile
\author{Alexander Rieppel} %Zwei Autoren
\title{Transportlogistik} %Titel/Thema
\subject{Betriebs- und Informationsmanagement} %Fach
\subtitle{Ausarbeitung} %Genaueres Thema, Optional
\date{\today} %Datum
\publishers{5AHITT} %Klasse

\maketitle
\tableofcontents


\chapter{Einführung}
	Der Begriff Transportlogistik beschreibt die Lieferung und Beförderung von Gütern, zwischen verschiedenen Orten innerhalb von Transportnetzwerken. Es beschreibt einen Teilbereich der Logistik, der durch die starken Differenzierungen, zwischen den Transportdienstleistern, sowie durch umweltpolitische Bestrebungen oder auch Stadt- oder Citylogistik-Projekten entstanden ist. Neben dem Transport von Waren als solches wird auch die Be- und Entladung zur Transportlogistik gezählt. \\\\ 
	Das Problem Güter von A nach B zu bringen ist für jeden Unternehmer eine Herausforderung. Prinzipiell ist es dabei egal, ob die Ware per LKW, Zug, Schiff, Flugzeug, Pipeline, etc. transportiert wird. Es geht rein darum, dass die Ware ankommt, sie möglichst unbeschadet ankommt und vor allem schnell am Ziel ist. Auch der Kostenfaktor spielt dabei eine große Rolle.\\\\
	Im Folgenden werden auch Fachbereiche behandelt, die besonders wichtig sind, in Verbindung mit der Transportlogistik. Darunter fallen Kapitel wie Transportnetzwerke, Güterverteilzentren, Standortplanung und auch die Wegplanung. \\\\
	
	\chapter{Güterverkehr}
	
	
	\chapter{Transportnetzwerke}
	\section{Einführung}
	ERP4.pdf - Grundstruktur von Transportnetzwerken (Anfang)
	
	\chapter{Güterverteilzentrum}
	\section{Allgemein}
	
	\section{Standortplanung}
	Standortplanung von Verteilzentren
	
	\chapter{Wegplanung}
	\section{Manhattan-Metrik}
	Manhattan Metrik\\
	ERP5.pdf - Materialwirtschaft Logistik
	\section{Französische Eisenbahnmetrik}
		Französische Eisenbahnmetrik

	\chapter{Supply-Chain-Management}
	\section{Was weiss ich}
	
\chapter{Quellen}
[1]\nolinkurl{URL} 
[2]\nolinkurl{http://de.wikipedia.org/wiki/Transportlogistik}
[3]\nolinkurl{}



\end{document}